\documentclass[12pt]{article}

\usepackage{tikz}
\usepackage{amsmath,amsfonts,amssymb}
\usepackage{graphicx}
\usepackage{hyperref}
\usepackage{listings}

\author{Aur\`ele Barri\`ere}
\title{LRC DM 1}
\date{February 3, 2017}

\def\exercise#1{\Large\textbf{Exercise #1}\normalsize\\}
\def\question#1{\textbf{Question #1:}\quad}


\begin{document}
\maketitle

\exercise{1}
\question{1} 
We show that the two Kripke models are bisimilar by giving a bisimulation relation between the two.
Let $R_{a1}, R_{b1}$ be the accessibility relations of the first Kripke model, and $R_{a2}, R_{b2}$ the accessibility relations of the second one.

Consider $Z$ the following binary relation :
$$\left\{
\begin{array}{l}
\forall n\in\mathbb{N},\quad w\ Z\ w^n \\
\forall n\in\mathbb{N},\quad v\ Z\ v^n
\end{array}
\right.$$

Let's show that the relation $Z$ is a bisimulation.
\paragraph{Atom} Every world in each Kripke model make true the same propositional letter  : $p$, and only this one. 

\paragraph{Forth} Let $n\in\mathbb{N}$. We have $w\ Z\ w^n$. The only successor of $w$ in the first Kripke model is $v$ : $w\ R_{a1}\ v$. We also have $w^n\ R_{a2}\ v^n$. And by definition of $Z$, $v\ Z\ v^n$.

Let $n\in\mathbb{N}$. We have $v\ Z\ v^n$. The only successor of $v$ in the first Kripke model is $w$ : $v\ R_{b1}\ w$. We also have $v^n\ R_{b2}\ w^{n+1}$. And by definition of $Z$, $w\ Z\ v^{n+1}$.

Thus, the \textit{Forth} rule is true for $Z$, as every pair in relation with $Z$ is either $w\ Z\ w^n$ or $v\ Z\ v^n$ by definition.

\paragraph{Back} Let $n\in\mathbb{N}$. We have $w\ Z\ w^n$. The only successor of $w^n$ in the first Kripke model is $v^n$. We also have $w\ R_{a1}\ v$ and $v\ Z\ v^n$. 

 Let $n\in\mathbb{N}$. We have $v\ Z\ v^n$. The only successor of $v^n$ in the first Kripke model is $w^{n+1}$. We also have $v\ R_{b1}\ w$ and $w\ Z\ w^{n+1}$. 

\paragraph{Conclusion} $Z$ is a non-empty bisimulation between the two Kripke models. The two Kripke models are thus bisimilar.


\question{2} The smallest Kripke model bisimilar to $\mathcal{M}$ must have at least one world, for the bisimulation to be non-empty.

The Kripke model with only one world in which the propositional letter $t$ is true is bisimilar to $\mathcal{M}$. Let $w$ be this world.

We then define $Z$ the bisimulation such that $9\ Z\ w$. $Z$ is a bisimulation : in both $9$ and $w$, only $t$ is true, and both $9$ and $w$ have no successors. And this Kripke model is the smallest possible as it only has one world.

The Kripke contraction is thus $(\{w\},\emptyset,V)$ such that $V(w)=\mathit{true}$.

\end{document}
